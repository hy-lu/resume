%% start of file `template-zh.tex'.
%% Copyright 2006-2013 Xavier Danaux (xdanaux@gmail.com).
%
% This work may be distributed and/or modified under the
% conditions of the LaTeX Project Public License version 1.3c,
% available at http://www.latex-project.org/lppl/.


\documentclass[12pt,a4paper,sans]{moderncv}   % possible options include font size ('10pt', '11pt' and '12pt'), paper size ('a4paper', 'letterpaper', 'a5paper', 'legalpaper', 'executivepaper' and 'landscape') and font family ('sans' and 'roman')

% moderncv 主题
% \moderncvstyle{classic}                        % 选项参数是 ‘casual’, ‘classic’, ‘oldstyle’ 和 ’banking’
% \moderncvcolor{blue}                          % 选项参数是 ‘blue’ (默认)、‘orange’、‘green’、‘red’、‘purple’ 和 ‘grey’
\moderncvstyle{banking}
% \moderncvtheme[blue]{classic}
% \nopagenumbers{}                             % 消除注释以取消自动页码生成功能

% 字符编码
\usepackage[utf8]{inputenc}                   % 替换你正在使用的编码
% \usepackage{CJKutf8}
\usepackage{xeCJK}
\setCJKmainfont{Noto Serif CJK SC}
\setCJKmonofont{Noto Sans Mono CJK SC}

% 调整页边距
% \usepackage[scale=0.85]{geometry}
\usepackage[margin=0.5in]{geometry}
\recomputelengths
\setlength{\hintscolumnwidth}{3cm}           % 如果你希望改变日期栏的宽度

% 对列表各项逆序编号(用于对文章进行编号)
\usepackage[itemsep=2pt]{etaremune}

% 个人信息
\name{Haoyang}{Lu}
% \title{简历题目 (可选项)}                     % 可选项、如不需要可删除本行
\address{Rm.1305, Wangkezhen Bldg., Peking University, Beijing}{}            % 可选项、如不需要可删除本行
\phone[mobile]{+8618670332855}              % 可选项、如不需要可删除本行
% \phone[fixed]{+2~(345)~678~901}               % 可选项、如不需要可删除本行
% \phone[fax]{+3~(456)~789~012}                 % 可选项、如不需要可删除本行
\email{luhy@pku.edu.cn}                    % 可选项、如不需要可删除本行
\social[github]{hy-lu}
\social[researchgate]{Haoyang\_Lu2}
% \social[linkedin]{haoyang-lu}
% \homepage{www.xialongli.com}                  % 可选项、如不需要可删除本行
% \extrainfo{附加信息 (可选项)}                 % 可选项、如不需要可删除本行
% \photo[64pt][0.4pt]{picture}                  % ‘64pt’是图片必须压缩至的高度、‘0.4pt‘是图片边框的宽度 (如不需要可调节至0pt)、’picture‘ 是图片文件的名字;可选项、如不需要可删除本行
% \quote{引言(可选项)}                          % 可选项、如不需要可删除本行

% 显示索引号;仅用于在简历中使用了引言
%\makeatletter
%\renewcommand*{\bibliographyitemlabel}{\@biblabel{\arabic{enumiv}}}
%\makeatother

% 分类索引
%\usepackage{multibib}
%\newcites{book,misc}{{Books},{Others}}
%----------------------------------------------------------------------------------
%            内容
%----------------------------------------------------------------------------------
\begin{document}
% \begin{CJK}{UTF8}{gbsn}                       % 详情参阅CJK文件包
\maketitle

\section{Research interest}
\cvitem{}{I am interested in computational psychiatry in general. My research involves using the methods of computational cognitive sciences to understand mental disorders. In my PhD thesis, I applied these methods to investigate the (non-)instrumental information sampling in autistic children. Currently, I am studying the cognitive processes involved in the formation and updating of superstitious beliefs. In the future, I hope to use AI and computational methods to further understand and help people with mental health issues.}

\section{Employment}
\cventry{2022 -- Present}{Postdoctoral Resaerch Fellow (with Boya postdoctoral fellowship)}{School of Psychological and Cognitive Sciences, Peking University}{}{}{Advisor: Hang Zhang}

\section{Education}
% \cventry{years}{degree/job title}{institution/employer} {localization}{optional: grade/...}{optional: comment/job description}
\cventry{2016 -- 2022}{PhD in Integrated Life Sciences (Psychology)}{Peking University}{Beijing, China}{}{}  % 第3到第6编码可留白
\cventry{2012 -- 2016}{BSc in Psychology}{Sun Yat-Sen University}{Guangzhou, China}{}{}

\section{Thesis}
% %to typeset lines with a hint on the left: 
% \cvline{leftmark}{text} or \cvitem{leftmark}{text}
\cvitem{Title}{\emph{Inference and Decision-making in People with Autism Spectrum Disorder or Broader Autism Phenotype}}
\cvitem{Advisors}{Li Yi, Hang Zhang}
\cvitem{Introduction}{\small The core features of autism spectrum disorders (ASD) are believed to be pertinent to how individuals interact with and sample the world. Therefore, it is crucial to understand the outcome of atypical active inference in ASD. In this interdisciplinary project, I worked with two advisors, one specializing in child psychopathology (Prof. Li Yi) and the other in computational cognitive science (Prof. Hang Zhang). We conducted a series of studies on information sampling in both autistic children and adults with a broader autism phenotype. Through the clinical experience with autistic children and the use of Bayesian linear mixed models and hierarchical Bayesian modeling, we gained a deeper understanding of the behavioral, attentional, and cognitive processes that differ between autistic individuals and neurotypical people in both instrumental and non-instrumental information sampling.}

\section{Publications}
\subsection{Peer-reviewed journal articles}
\newcommand{\Revision}{\textit{Under revision}}
\newcommand{\Review}{\textit{Under review}}
\newcommand{\Submitted}{\textit{Submitted}}
\newcommand{\Manuscript}{\textit{Preparing manuscript}}
\newcommand{\CS}{*} % corresponding author
\newcommand{\CF}{\textsuperscript{\#}} % co-first author
\newcommand{\ME}{\textbf{Lu, H.}}

% \CS corresponding author, \CF co-first author.

% \begin{enumerate}
\begin{etaremune}
  \item Wei, N., \ME, Wang, Q., Song, C., Yi, L. (2023). Vigilance or avoidance: How do autistic traits and social anxiety modulate attention to the eyes? Frontiers in Neuroscience, 16, 1081769.
  \item Hu, Y., Xiong, Q., Wang, Q., Song, C., Wang, D., \ME, Shi, W., Han, Y., Liu, J., Li, X., \& others. (2022). Early development of social attention in toddlers at high familial risk for autism spectrum disorder. Infant Behavior and Development, 66, 101662.
  \item Wang, Q., \ME, Feng, S., Song, C., Hu, Y., \& Yi, L. (2021). Investigating intra-individual variability of face scanning in autistic children. Autism : The International Journal of Research and Practice, 13623613211064372.
  \item Feng, S., \ME, Wang, Q., Li, T., Fang, J., Chen, L., \& Yi, L. (2021). Face-viewing patterns predict audiovisual speech integration in autistic children. Autism Research.
  \item Feng, S., \ME, Fang, J., Li, X., Yi, L., \& Chen, L. (2021). Audiovisual speech perception and its relation with temporal processing in children with and without autism. Reading and Writing, 1–22.
  \item \ME, Yi, L., \& Zhang, H. (2019). Autistic traits influence the strategic diversity of information sampling: Insights from two-stage decision models. PLoS Computational Biology, 15(12), e1006964.
  \item \ME, Li, P., Fang, J., \& Yi, L. (2019). The perceived social context modulates rule learning in autism. Journal of Autism and Developmental Disorders, 49(11), 4698–4706.
  \item Zhang, Y., Song, W., Tan, Z., Zhu, H., Wang, Y., Lam, C. M., Weng, Y., Hoi, S. P., \ME, Chan, B. S. M., \& others. (2019). Could social robots facilitate children with autism spectrum disorders in learning distrust and deception? Computers in Human Behavior, 98, 140–149.
  \item Li, T., Hu, Y., Song, C., \ME, \& Yi, L. (2018). The measurements and mechanisms of restricted and repetitive behaviors in autism spectrum disorders. Chinese Science Bulletin, 63(15), 1438–1451.
  \item Yang, Y., Tian, Y., Fang, J., \ME, Wei, K., \& Yi, L. (2017). Trust and deception in children with autism spectrum disorders: A social learning perspective. Journal of Autism and Developmental Disorders, 47(3), 615–625.

% \end{enumerate}
\end{etaremune}

\subsection{Conference abstracts}
% \begin{enumerate}
\begin{etaremune}
  \item \ME, Teng, T., \& Zhang, H. (2023). The formation of superstitions in an uncontrollable environment. Proceedings of the Annual Meeting of the Cognitive Science Society, 45. Retrieved from https://escholarship.org/uc/item/5fx4t61x
  \item \ME, Yi, L., \& Zhang, H. (2022). Adults with more autistic traits are more willing to pay for “useless” information. INSAR 2022 Annual Meeting.
  \item \ME, Yi, L., \& Zhang, H. (2020). Inefficient information sampling under explicit costs in children with ASD. INSAR 2020 Virtual Meeting.
  \item Song, C., Wang, Q., Xu, J., \ME, Qin, S., \& Yi, L. (2020). Baseline arousal modulates face scanning in autism spectrum disorder. INSAR 2020 Virtual Meeting.
  \item \ME, Zhang, H., Yi, L. (2018). Adults with high autistic traits are reluctant to trade accuracy for monetary reward: a probabilistic reasoning experiment. INSAR 2018 Annual Meeting.
  \item \ME, Li, P., Yi, L. (2017) Impaired Rule Learning in Social Context of Children with Autism
% \end{enumerate}
\end{etaremune}



\section{Teaching}
\cventry{2018 -- 2023}{Workshop lecturer}{R for Eye-tracking data analysis}{Chongqing, Nanjing, Shanghai, Beijing}{}{Design and deliver a 2-day workshop for learning to use R for data analysis, particularly eye-tracking data.}
\cventry{2021 -- 2023}{Teaching assistant}{Effective writing and communication in science}{Peking University}{}{Design, tutorial delivery, and marking. Also deliver a 1\textasciitilde2 hr lecture on how to do scientific data visualization.}
\cventry{2020}{Guest lecturer}{Introduction to Cognitive Modeling}{Peking University}{}{Delivery a one-hour tutorial on Reproducibility and Literate Programming in R}
\cventry{2018}{Teaching assistant}{Child psychopathology}{Peking University}{}{Marking and organizing group projects.}
\cventry{2017}{Teaching assistant}{Topics in Autism Research}{Peking University}{}{Design, marking, and organizing class discussion.}

\section{Professional membership and service}
\cvitem{Membership}{Society for Research in Child Development; International Society for Autism Research; Cognitive Science Society; Society for Neuroeconomics}
\cvitem{Reviewing}{Autism Research; Journal of Autism and Developmental Disorders; eLife; OpenMind}

\section{Professional skills and languages}
\cvitemwithcomment{Research}{Eye-tracking, computational modeling}{Expert}
\cvitemwithcomment{}{fNIRS, fMRI, EEG}{Beginner}
\cvitemwithcomment{Statistics}{Bayesian statistics, Generalizaed linear mixed model}{Expert}
\cvitemwithcomment{}{Generalized additive model, Survival analysis, Structural equation models}{Proficient}
\cvitemwithcomment{}{Machine learning methods}{Intermediate}
\cvitemwithcomment{Programming}{R, MATLAB, Psychotoolbox}{Expert}
\cvitemwithcomment{}{Stan, SPSS, PsychoPy/PsychoJs, Git}{Proficient}
\cvitemwithcomment{}{Python, JAGS, Mplus, E-Prime, Visual Basic, C++, \LaTeX}{Intermediate}
\cvitemwithcomment{Languages}{English}{Advanced}
\cvitemwithcomment{}{Chinese}{Native}

\clearpage
% \end{CJK}
\end{document}


%% 文件结尾 `template-zh.tex'.
